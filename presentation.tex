% © 2013 Saurabh Kumar
%
% use xelatex %<
%
\documentclass[xetex,serif,compress]{beamer}
\usepackage{fontspec}
\usepackage{xunicode} % Unicode extras!
\usepackage{xltxtra}  % Fixes
\usepackage{listings}
\usepackage{fancyvrb}
\setmainfont{Trebuchet MS}
\setmonofont{Inconsolata}
\usetheme{default}

\setbeamertemplate{frametitle}{
  \color{black}
  \vspace*{0.5cm}
  \hspace*{0.25cm}
  \textbf{\insertframetitle}
  \par
}

% Hide the navigation icons at the bottom of the page
\setbeamertemplate{navigation symbols}{}

% No margins on any side
\setbeamersize{text margin left=0cm,text margin right=0cm}


\begin{document}

% slide with bullet points
\newcommand{\mslide}[2]{
  \begin{frame}{#1}
    \begin{center}
      \begin{list}{$\bullet$}{\itemsep=1em}
        #2
      \end{list}
    \end{center}
  \end{frame}
}

\frame{
  \begin{center}
    \vspace{1.5cm}
           {\huge Unix}\\
           {\large Philosophy and Workflow}\\
           \vspace{3cm}
           Saurabh Kumar\\
           \vspace{0.5cm}
           2013-03-10\\
  \end{center}
}

\mslide{40+ years of Unix}{
\item Unix was written in 1969.
\item A unique culture and philosophy has evolved around it.
  % Original unix utilities cat, grep, sed, still used.
\item Small, correct programs that do one thing well.
  % If you need to do a new job build a new program, rather than making a complicated beast.
\item Programs should work together.
  % Text is a universal interface.
\item Programs should handle text, because text is a universal interface.
}

\begin{frame}{}
  \begin{center}
    \huge
    Simple != Easy\\
    \vspace{2cm}
    \large
    A learning curve is involved.
  \end{center}
\end{frame}

\mslide{RTFM}{
\item Man pages are your friend.
\item Try it out. \texttt{man cat, man less, man tar}
\item Man pages are divided into sections.
\item You can search for man pages using apropos.
\item \texttt{apropos strlen, man 3 strlen}
}


\mslide{CLI}{
\item A command line interface is very powerful.
\item \texttt{ls, mkdir foo, cd foo, touch file, cp file .., mv ../file .}
\item \texttt{echo ``Hello'' >> file, cat file}
\item \texttt{for i in `seq 1 100`;\\
  do\\
  echo "Hello"\$i >>file\\
  done}
\item \texttt{cat file, cat file | less}
}

\mslide{Concepts}{
\item (Nearly)Every thing is a file.
\item Your harddisk is \texttt{/dev/sda-z}
\item Standard input, output are \texttt{/dev/stdin, /dev/stdout}
\item Piping and redirection. \texttt{|, <, >, >>}
\item \texttt{cat /dev/stdin, cat /dev/stdin >> file, cat /dev/stdin > /dev/stdout}
\item Burning a usb image: \texttt{cat debian.iso > /dev/sdb \&\& sync}
}

\mslide{Other examples}{
\item Download input file from \url{http://sprunge.us/cdGI} or from \url{https://raw.github.com/sa1/bisect-demo/master/3lpigs}
\item \texttt{tr ' ' '\textbackslash{}n' < 3lpigs | sort\\ tr ' ' '\textbackslash{}n' < 3lpigs | sort | uniq -c | sort -n}
\item \texttt{grep 'wolf' 3lpigs}
\item \texttt{sed -i 's/wolf/wolves/g' 3lpigs}
\item Wildcards: \texttt{ls fi*, ls *.*}
}

\mslide{Automate the boring parts}{
\item Script what needs to be done.
\item \texttt{\#! /usr/bin/bash\\
 echo ``Running task'' >> file
}
\item \texttt{at 0430\\
> notify-send ``Executing task''\\
> ./script.sh}
\item Use \texttt{cron} for scheduling periodic tasks.
\item Startup scripts - \texttt{systemd, /etc/profile.d/, \textasciitilde{}/.xinitrc}
}

\frame{
  \begin{center}
    \vspace{1.5cm}
           {\large Developer workflow}\\
  \end{center}
}

\mslide{Unix as your IDE}{
\item A single large IDE vs a large number of tools that work together.
\item Edit - \texttt{vim, emacs}
\item Build - \texttt{make}
\item Debug - \texttt{gdb, valgrind, strace, lsof}
\item Version Control - \texttt{git, svn, diff, patch}
}

\mslide{Editors}{
\item Vim and Emacs
\item Syntax highlighting, modes, macros
\item Extensibility, plugins - lint, syntax checkers.
\item Communicating with external programs.
}

\begin {frame}[fragile]
  \frametitle{Make}
%\item It's a build recipe.
\begin{verbatim}
            CC = gcc
            OBJECTS = main.o example.o library.o
            BINARY = example

            all: example

            example: $(OBJECTS)
                $(CC) $(OBJECTS) -o $(BINARY)

            docs: docs.html

            docs.html: docs.md
                markdown docs.md > docs.html

            clean:
                rm -f $(BINARY) $(OBJECTS)

            install: example
                cp $(BINARY) /usr/bin
\end{verbatim}
\end{frame}

\mslide{Testing}{
\item Use scripts to set up the environment.
\item Scripts to test a system.
\item Testsuites
\item CI- Continuous Integration
\item Build, Compile, Test, Fix, Commit.
}

\mslide {Version Control}{
\item file1.c, file1final.c, file1\_realfinal.c
\item \texttt{git init, echo ``Hello'' >> file, git add file, git commit}
\item \texttt{echo ``Hello 2'' >> file, git status, git diff, git add file, git commit}
\item \texttt{git log}
\item Branches, diffs, patches
}

\mslide {Git Bisect}{
\item Using binary search to find buggy commits.
\item \texttt{git clone https://github.com/sa1/bisect-demo.git}
\item \texttt{git log, git bisect start}
\item \texttt{git bisect bad} to mark bad commits
\item \texttt{git bisect good} to mark good commits.
\item \texttt{git bisect reset} to reset.
}

\mslide{Thank you}{
\item Feedback
\item \url{http://klug.github.com}
}

\end{document}
